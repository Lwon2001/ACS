\documentclass[letterpaper]{seminar}

\scrollmode

\slideframe{none}

%\setcounter{chapter}{0}

\usepackage{epsfig,amsmath,amsfonts,euscript,times,harvard,color,float}

%\usepackage{epsfig,amsmath,amsfonts,euscript,times,harvard,color,fancyhdr,float}

\def\bias {{\rm Bias}}

\def\ev {{\rm E}}

\def\EPE{{\rm EPE}}

\def\argmax {{\rm argmax}}

\def\Ave{{\rm Ave}}

\def\cov {{\rm Cov}}

\def\var {{\rm Var}}

\def\mse {{\rm Mse}}

\def\bX {{\bf X}}

\def\bx {{\bf x}}

\def\Pr{{\rm Pr}}

%\renewcommand{\headrulewidth}{0pt}

\textwidth=9in

%\hypersetup{backref,%

% pdfpagemode=text,%

% co%lorlinks=true,

% linkcolor=green,

%baseurl=http://www-stat.stanford.edu/\string~hastie/Printer/315-LECTURES/%

%}

%\pagestyle{fancy}
\def\printlandscape{\special{landscape}}

\begin{document}

%\newcommand{\RED}{\textcolor{red}{RED}}

%\newcommand{\GREEN}{\textcolor[rgb]{0,.8,0}{GREEN}}

%\newcommand{\gem}{\color[rgb]{0,.8,0}\em}

%\def\RED{\Co{\color{red}RED}}

%\def\GREEN{\Co{\color{green}GREEN}}

\newcommand{\gem}{}

\def\RED{}

\def\GREEN{}

\thispagestyle{empty}

\begin{slide}
\centerline{\bf More details  of K-means clustering}

\end{slide}
\begin{slide}

{\bf K-means clustering algorithm}

\begin{enumerate}

\setcounter{enumi}{-1}

\item Start with initial guesses for cluster centers (centroids)

\item For each data point, find closest cluster center (partitioning step)

\item Replace each centroid by average of data points in its partition

\item Iterate 1+2 until convergence

\end{enumerate}

(See Fig 14.4, 14.6) \\

Write $x_i = (x_{i1}, ... x_{ip})$:

If centroids are $m_1, m_2, ... m_k$, and partitions are

$c_1, c_2, ... c_k$, then one can show that K-means converges to a {\it local} minimum of

\[
\sum^K_{k=1} \sum_{i\in c_k} || x_i - m_k ||^2 \ \ \ \ \ \ \ \rm Euclidean \ distance
\]

(within cluster sum of squares) \\

{\bf In practice:}

\begin{itemize}

\item Try many random starting centroids (observations) and choose solution with smallest of squares

\end{itemize}

{\bf How to choose K?}

\begin{itemize}

\item Difficult -- details later

\end{itemize}

\end{slide}
\begin{slide}

{\bf Stepping back}

\begin{itemize}

\item All clustering algorithms start with a dissimilarity measure for $j^{th}$ feature

$$d_j (x_{ij} , x_{i' j}) \ \rm and\ define $$

$$D (x_i, x_{i'}) = \sum^P_{j=1} d_j (x_{ij}, x_{{i'}j} ) $$

\item[] Usually $d_j (x_{ij}, x_{i' j}) = (x_{ij} - x_{i'j})^2$

\end{itemize}

\newpage
\end{slide}
\begin{slide}

{\bf Other possibilities:}

\begin{itemize}

\item Correlation

$$\rho (x_i, x_{i'}) =
\displaystyle\frac {\sum_j (x_{ij} - \overline x_i) (x_{i' j} - \overline x_{i'} )}
{ \sqrt{\sum_j (x_{ij} - \overline x_i )^2 \sum_j (x_{i'j} - \overline x_{i'})^2 }}$$

\item[ ] $\overline x_i$ = mean of observation $i$ 

\item If observations are standardized:

$$x_{ij} \leftarrow \frac{ x_{ij} - \bar x_{i} }
{\sqrt{ \sum_j (x_{ij} - \bar x_i)^2} } $$

\item[] then $2(1 - \rho (x_i, x_{i'} )) = \sum_j (x_{ij} - x_{i, j'})^2$
\item[] So clustering via correlation $\equiv$ clustering via Euclidean distance with standardized features

\end{itemize}

\newpage


{\bf Partitioning (Clustering) Algorithms}

\begin{itemize}

\item Group assignment function (``encoder'') $C(i)$

\item[] $C: 1, 2, ... N \rightarrow (1, 2, ...K)$

\item {\bf Criterion}: choose $C$ to minimize

$$W (C) = {1\over 2} \sum^K_{k=1} \sum_{C(i)=k} \sum_{C({i'})=k} d (x_i , x_{i'} ) $$ %your note had "D" in upper case

\item[] (within cluster scatter)

\end{itemize}

\newpage

{\bf Fact}:

\begin{itemize}
\item $K$-means minimizes $W(C)$ when $D = || x_i - x_{i'} ||^2$
\begin{eqnarray*}
W(C) & = &{1\over 2} \sum^K_{k=1} \sum_{C(i)=k} \sum_{C(i')=k} || x_i - x_{i'} ||^2 \\
& = &\sum^K_{k=1} N_k \sum_{C(i)=k} || x_i - \overline x_k ||^2 %your note didn't have N_k in the middle
\end{eqnarray*}

\item K-means solves {\it enlarged} problem:

$$ \min_{C, m_1 ... m_k} \sum_k \sum_{C(i)=k} ||x_i - m_k ||^2 $$ %should N_k be inserted in between \sum

\item[] to find assignment function $C$

\end{itemize}

\end{document}